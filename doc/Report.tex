\documentclass[11pt, a4paper]{article}

\usepackage[utf8]{inputenc}
\usepackage{fullpage}
\usepackage{graphicx}
\graphicspath{ {images/} }
\usepackage{listings}
\usepackage{mathtools}
\usepackage{amssymb}
\usepackage{eurosym} %Euro symbol
%\usepackage[ngerman]{babel}
\usepackage{cancel} %Kürzen

\newcommand\braces[1]{\left(#1\right)}
\newcommand\brackets[1]{\left[#1\right]}
\renewcommand{\vec}[1]{\underline{#1}}
\newcommand{\mat}[1]{\underline{\underline{#1}}}
\newcommand{\abs}[1]{\left\lvert#1\right\rvert}
\newcommand{\norm}[1]{\left\lVert#1\right\rVert}
\newcommand\tr[1]{\mathrm{tr}\br{#1}}
\newcommand\average[1]{\left\langle#1\right\rangle}
\newcommand{\acos}[1]{\mathrm{acos}\braces{#1}}
\newcommand{\asin}[1]{\mathrm{asin}\braces{#1}}
\newcommand{\dx}[1][x]{\ \mathrm{d#1}}
\newcommand\expectedValue[1]{\mathbb{E}\braces{#1}}
\newcommand\variance[1]{\mathbb{V}\braces{#1}}
\newcommand\setequal{\overset{!}{=}}
\newcommand{\gerquote}[1]{\glqq#1\grqq}

\title{Algorithmische Geometrie Projekt \\ Trapezzerlegung}
\author{Yannick Schrör \and Christian Mielers}
\date{\today}

% Warum in Python?

\begin{document}
\maketitle

\section{Programmnutzung}
Bei den Programmen zur Trapezzerlegung und Pfadsuche handelt es sich um Python3-Programme. Daher muss auf der Zielplattform eine Python3-Runtime installiert sein, das ältere Python2 wird nicht unterstützt. Darüber hinaus weisen die Programme keine externen Abhängigkeiten (etwa zu Drittbibliotheken) auf.

\paragraph{Trapezzerlegung} Die Trapezzerlegung wird in der Datei \texttt{trapezoid\_decomposition.py} realisiert. Die Syntax zum Programmaufruf lautet

trapezoid\_decomposition.py [-h] [-i] [-d] [-l] in\_file out\_file

\begin{tabular}{|c|c|c|}
	\hline
	Parameter & erforderlich & Bedeutung \\
	\hline
	in\_file & * & Der Pfad zur Eingabedatei \\
	out\_file & * & \\
	h & & \\
	i & & \\
	d & & \\
	l & & \\
	\hline
\end{tabular}

\paragraph{Pfadsuche}

\end{document}










