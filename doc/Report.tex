\documentclass[11pt, a4paper]{article}

\usepackage[utf8]{inputenc}
\usepackage{fullpage}
\usepackage{graphicx}
\graphicspath{ {images/} }
\usepackage{listings}
\usepackage{mathtools}
\usepackage{amssymb}
\usepackage{eurosym} %Euro symbol
%\usepackage[ngerman]{babel}
\usepackage{cancel} %Kürzen
\usepackage{pbox}

\newcommand\braces[1]{\left(#1\right)}
\newcommand\brackets[1]{\left[#1\right]}
\renewcommand{\vec}[1]{\underline{#1}}
\newcommand{\mat}[1]{\underline{\underline{#1}}}
\newcommand{\abs}[1]{\left\lvert#1\right\rvert}
\newcommand{\norm}[1]{\left\lVert#1\right\rVert}
\newcommand\tr[1]{\mathrm{tr}\br{#1}}
\newcommand\average[1]{\left\langle#1\right\rangle}
\newcommand{\acos}[1]{\mathrm{acos}\braces{#1}}
\newcommand{\asin}[1]{\mathrm{asin}\braces{#1}}
\newcommand{\dx}[1][x]{\ \mathrm{d#1}}
\newcommand\expectedValue[1]{\mathbb{E}\braces{#1}}
\newcommand\variance[1]{\mathbb{V}\braces{#1}}
\newcommand\setequal{\overset{!}{=}}
\newcommand{\gerquote}[1]{\glqq#1\grqq}

\title{Algorithmische Geometrie Projekt \\ Trapezzerlegung}
\author{Yannick Schrör \and Christian Mielers}
\date{\today}

% Warum in Python?

\begin{document}
\maketitle

\section{Performance}
\begin{tabular}{|c|c|c|c|}
	\hline
	m & Decomposition & Face assignment & Point grouping \\
	\hline
\end{tabular}

\section{Programmnutzung}
Bei den Programmen zur Trapezzerlegung und Pfadsuche handelt es sich um Python3-Programme. Daher muss auf der Zielplattform eine Python3-Runtime installiert sein, das ältere Python2 wird nicht unterstützt. Darüber hinaus weisen die Programme keine externen Abhängigkeiten (etwa zu Drittbibliotheken) auf.

\paragraph{Trapezzerlegung} Die Trapezzerlegung wird in der Datei \texttt{trapezoid\_decomposition.py} realisiert. Die Syntax zum Programmaufruf lautet

\begin{quotation}
	\texttt{python3 trapezoid\_decomposition.py [-h] [-i] [-d] [-l] in\_file out\_file}
\end{quotation}

\begin{tabular}{|l|c|l|}
	\hline
	Parameter & erforderlich & Bedeutung \\
	\hline
	in\_file & * & \pbox{10cm}{Der Pfad zur Eingabedatei, in der die Punkte, Strecken und Abfragepunkte gespeichert sind} \\
	out\_file & * & Der Pfad zur Datei, in die die Ausgabe geschrieben wird \\
	h & & Hilfe zum Programmaufruf anzeigen \\
	i & & Die Eingabedaten (\textbf{i}nput) visualisieren \\
	d & & Die Zerlegung (\textbf{d}ecomposition) visualisieren \\
	l & & Lokalisierungsergebnisse (\textbf{l}ocalization) visualisieren \\
	\hline
\end{tabular}

Die Visualisierungsoptionen können alle im selben Aufruf genutzt werden.

\paragraph{Pfadsuche} Die Pfadsuche mittels Trapezzerlegung wird in der Datei \texttt{path\_finding.py} realisiert. Die Syntax zum Programmaufruf lautet

\begin{quotation}
	\texttt{python3 path\_finding.py [-h] [-d] [-r] [-p] in\_file out\_file}
\end{quotation}

\begin{tabular}{|l|c|l|}
	\hline
	Parameter & erforderlich & Bedeutung \\
	\hline
	in\_file & * & \pbox{10cm}{Der Pfad zur Eingabedatei, in der die Hindernisse und Start- sowie End-punkte der zu findenden Pfade gespeichert sind} \\
	out\_file & * & \pbox{10cm}{Der Pfad zur Datei, in die die gefundenen Pfade geschrieben werden} \\
	h & & Hilfe zum Programmaufruf anzeigen \\
	d & & \pbox{10cm}{Die Trapezzerlegung (\textbf{d}ecomposition) der Eingabedaten visualisieren, nachdem die Hindernis-Trapeze entfernt wurden} \\
	r & & Die \textbf{R}oad-map visualisieren \\
	p & & Die gefundenen Pfade (\textbf{p}aths) visualisieren \\
	\hline
\end{tabular}

Die Visualisierungsoptionen können alle im selben Aufruf genutzt werden.

\end{document}










